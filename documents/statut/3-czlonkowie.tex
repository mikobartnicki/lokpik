\section{Członkowie Klubu}
\begin{enumerate}
    \item Członkowie Klubu dzielą się na:
    \begin{enumerate}
        \item członków zwyczajnych;
        \item członków wspierających;
        \item członków honorowych.
    \end{enumerate}
    \item Każda osoba, która wzięła czynny udział w~zebraniu założycielskim, staje się członkiem zwyczajnym oraz otrzymuje tytuł Założyciela Klubu.
    \item Tytuł Założyciela Klubu jest niezbywalny i~nie może zostać nadany ani uzyskany w~żaden inny sposób, jak tylko poprzez czynny udział w~zebraniu założycielskim Klubu.
    \item Członkiem zwyczajnym może zostać osoba fizyczna o~nieposzlakowanej opinii.
    \item Członkiem wspierającym lub członkiem honorowym może zostać osoba fizyczna lub osoba prawna.
    \item Członków zwyczajnych oraz członków wspierających przyjmuje Zarząd, na podstawie złożonej deklaracji członkowskiej.
    \item Osoba fizyczna lub osoba prawna może zostać członkiem honorowym tylko na zaproszenie Zarządu.
    \item Wszyscy członkowie Klubu mają obowiązek:
    \begin{enumerate}
        \item przestrzegania postanowień statutu i~uchwał Zarządu;
        \item aktywnego uczestnictwa w~realizacji celów Klubu (nie dotyczy członków wspierających i~członków honorowych);
        \item terminowego opłacania składek członkowskich (nie dotyczy członków honorowych).
    \end{enumerate}
    \item Członkowie wspierający oraz członkowie honorowi, zwolnieni są z~obowiązku aktywnego uczestnictwa w~realizacji celów Klubu.
    \item Członkowie honorowi zwolnieni są z~obowiązku opłacania składek członkowskich.
    \item Członkowie zwyczajni mają prawo do:
    \begin{enumerate}
        \item korzystania na równych zasadach z~dorobku i~wszelkich form aktywności Klubu;
        \item zgłaszania wniosków dotyczących działalności Klubu;
        \item czynnego uczestnictwa w~Walnych Zebraniach;
        \item czynnego i~biernego uczestnictwa w~wyborach do władz Klubu.
    \end{enumerate}
    \item Członkowie wspierający oraz członkowie honorowi mają prawo do:
    \begin{enumerate}
        \item korzystania z~dorobku i~wszelkich form aktywności Klubu w miarę ich dostępności;
        \item zgłaszania wniosków dotyczących działalności Klubu;
        \item biernego uczestnictwa w~Walnych Zebraniach.
    \end{enumerate}
    \item Członek zwyczajny lub członek wspierający, który nie opłacił w~wymaganym terminie składki członkowskiej, jest zawieszony w~swoich prawach, do czasu uregulowania rzeczonej zaległości.
    \item Członkostwo w~Klubie ustaje w~następujących przypadkach:
    \begin{enumerate}
        \item dobrowolnej rezygnacji;
        \item wykluczenia decyzją Zarządu;
        \item śmierci.
    \end{enumerate}
    \item Zarząd może wykluczyć członka w~następujących przypadkach:
    \begin{enumerate}
        \item łamania statutu lub uchwał Zarządu;
        \item nieprzestrzegania zasad bezpieczeństwa na strzelnicy;
        \item niegodnego zachowania;
        \item zachowania naruszającego dobre imię Klubu;
        \item aktywności sprzecznej z~celami działalności Klubu;
        \item braku aktywności zgodnej z~celami działalności klubu (nie dotyczy członków wspierających oraz członków honorowych);
        \item dłuższej niż 60 dni zaległości w~opłaceniu składki członkowskiej (nie dotyczy członków honorowych).
    \end{enumerate}
\end{enumerate}