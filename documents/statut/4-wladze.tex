\section{Władze Klubu}
\begin{enumerate}
    \item Organami władzy Klubu są:
    \begin{enumerate}
        \item Walne Zebranie;
        \item Zarząd;
        \item Komisja Rewizyjna.
    \end{enumerate}
    \item Członkowie organów władzy Klubu pełnią swoje funkcje na zasadzie pracy społecznej, nie pobierając z~tego tytułu wynagrodzenia ani żadnych dodatkowych korzyści ze strony Klubu.
\end{enumerate}
\subsection{Walne Zebranie}
\begin{enumerate}
    \item Rolą Walnego Zebrania jest:
    \begin{enumerate}
        \item uchwalanie zmian treści statutu Klubu;
        \item powoływanie i~odwoływanie wybieralnych władz Klubu;
        \item rozpatrywanie wniosków zgłoszonych przez członków Klubu.
    \end{enumerate}
    \item Walne Zebranie jest zwoływane przez Prezesa w~następujących przypadkach:
    \begin{enumerate}
        \item na wniosek Zarządu;
        \item na wniosek Komisji Rewizyjnej;
        \item na wniosek co najmniej 2/3 łącznej liczby członków zwyczajnych Klubu.
    \end{enumerate}
    \item Walne Zebranie musi zostać zwołane najpóźniej w~ciągu 3~miesięcy od zgłoszenia wniosku o~jego zwołanie.
    \item Datę i miejsce Walnego Zebrania, Prezes podaje do wiadomości członków Klubu, z~wyprzedzeniem nie mniejszym niż 3~dni.
    \item Walne Zebranie podejmuje decyzje na drodze głosowania jawnego, ważoną większością głosów, przy obowiązkowym udziale Prezesa, niezależnie od liczby obecnych osób uprawnionych do głosowania, stosując następujące wagi głosów:
    \begin{enumerate}
        \item głos członka zwyczajnego, pełniącego funkcję w~Zarządzie i~posiadającego tytuł Założyciela Klubu, ma wagę 50~(pięćdziesiąt);
        \item głos członka zwyczajnego, pełniącego funkcję w~Zarządzie i~nieposiadającego tytułu Założyciela Klubu, ma wagę 10~(dziesięć);
        \item głos członka zwyczajnego, niepełniącego funkcji w~Zarządzie i~posiadającego tytuł Założyciela Klubu, ma wagę 5~(pięć);
        \item głos członka zwyczajnego, niepełniącego funkcji w~Zarządzie i~nieposiadającego tytułu Założyciela Klubu, ma wagę 1~(jeden).
    \end{enumerate}
    \item Członkowie wspierający oraz członkowie honorowi mogą uczestniczyć biernie w Walnym Zebraniu, stanowiąc głos doradczy, lecz nie biorąc udziału w~głosowaniach.
\end{enumerate}
\subsection{Zarząd}
\begin{enumerate}
    \item Rolą Zarządu jest:
    \begin{enumerate}
        \item kierowanie działalnością Klubu;
        \item organizowanie aktywności statutowych Klubu;
        \item przyjmowanie nowych członków Klubu.
    \end{enumerate}
    \item W~skład Zarządu wchodzą:
    \begin{enumerate}
        \item Prezes, odpowiedzialny za:
        \begin{enumerate}
            \item kierowanie pracą Zarządu;
            \item prowadzenie rejestru członków Klubu;
            \item zastępstwo pozostałych członków Zarządu, w~przypadku nastania takiej potrzeby;
            \item zwoływanie Walnego Zebrania w przypadkach określonych niniejszym statutem.
        \end{enumerate}
        \item Skarbnik, odpowiedzialny za:
        \begin{enumerate}
            \item nadzór nad finansami Klubu;
            \item ewidencję składek członkowskich i~opłat;
            \item sporządzanie raportów finansowych Klubu;
        \end{enumerate}
        \item Sekretarz, odpowiedzialny za:
        \begin{enumerate}
            \item prowadzenie archiwum dokumentacji Klubu;
            \item protokołowanie Walnych Zebrań oraz zebrań Zarządu.
        \end{enumerate}
    \end{enumerate}
    \item Skład pierwszego Zarządu wybierają spośród siebie Założyciele Klubu podczas zebrania założycielskiego.
    \item Skład Zarządu może być zmieniony tylko decyzją Walnego Zebrania.
    \item Zarząd podejmuje decyzje podczas zebrań, na drodze głosowania jawnego, bezwzględną większością równoważnych głosów, przy obowiązkowym udziale większości członków Zarządu, w tym Prezesa.
    \item W~przypadku nierozstrzygnięcia głosowania podczas zebrania Zarządu, decyduje głos oddany przez Prezesa.
\end{enumerate}
\subsection{Komisja Rewizyjna}
\begin{enumerate}
    \item Rolą Komisji Rewizyjnej jest:
    \begin{enumerate}
        \item kontrolowanie prawidłowości funkcjonowania Klubu;
        \item kontrolowanie prawidłowości pracy Zarządu.
    \end{enumerate}
    \item Komisja Rewizyjna składa się z~trzech członków Klubu, którzy wybierają spośród siebie Przewodniczącego.
    \item Członkowie Komisji Rewizyjnej nie mogą być członkami Zarządu ani pozostawać z~nimi w~związku małżeńskim, we wspólnym pożyciu, w~stosunku pokrewieństwa, powinowactwa lub podległości służbowej.
    \item Skład pierwszej Komisji Rewizyjnej wybierają spośród siebie Założyciele Klubu podczas zebrania założycielskiego.
    \item Skład Komisji Rewizyjnej może być zmieniony jedynie decyzją Walnego Zebrania.
    \item Komisja Rewizyjna podejmuje decyzje podczas zebrań, na drodze głosowania jawnego, bezwzględną większością równoważnych głosów, przy obowiązkowym udziale większości członków Komisji Rewizyjnej, w~tym Przewodniczącego.
    \item W~przypadku nierozstrzygnięcia głosowania podczas zebrania Komisji Rewizyjnej, decyduje głos oddany przez Przewodniczącego.
\end{enumerate}